\section{The Derivative}
\begin{definition}
	\label{definition-4.2.1}
	Let $I \subseteq \mathbb{R}$ be an open interval, let $c \in I$, and let $f~:~I~\to~\mathbb{R}$ be a function.
	\begin{enumerate}
		\item The function $f$ is \textbf{differentiable} at $c$ if
		\begin{equation}
		\label{equation-definition-4.2.1}
		\lim\limits_{x \to c} \frac{f(x) - f(c)}{x - c}
		\end{equation}
		exists; if this limit exists, it is called the \textbf{derivative} of $f$ at $c$, and it is denoted $f^\prime (c)$.
		\item The function $f$ is \textbf{differentiable} if it is differentiable at every number in $I$. If $f$ is differentiable, the \textbf{derivative} of $f$ is the function $f^\prime : I \to \mathbb{R}$ whose value at $x$ is $f^\prime (x)$ for all $x \in I$.
	\end{enumerate}
\end{definition}

\begin{lemma}
	\label{lemma-4.2.2}
	Let $I \subseteq \mathbb{R}$ be an open interval, let $c \in I$, and let $f:I\to\mathbb{R}$ be a function. Then $f$ is differentiable at $c$ if and only if
	\begin{equation}
	\label{equation-lemma-4.2.2}
	\lim\limits_{h \to 0} \frac{f \left( c + h \right) - f \left( c \right)}{h}
	\end{equation}
	exists, and if this limit exists it equals $f^\prime \left( c \right)$.
\end{lemma}

\subsection{Exercises}
\begin{exercise}
	\label{exercise-4.2.1}
	Using only the definition of derivatives and Lemma~\ref{lemma-4.2.2}, find the derivative of each of the following functions.
	\begin{enumerate}
		\item Let $f:\mathbb{R}\to\mathbb{R}$ be defined by $f(x)=3x-8$ for all $x\in\mathbb{R}$.
		\item Let $g:\mathbb{R}\to\mathbb{R}$ be defined by $g(x)=x^3$ for all $x\in\mathbb{R}$.
	\end{enumerate}
\end{exercise}
