
\documentclass{book}

\usepackage[utf8]{inputenc}
\usepackage[T1]{fontenc}
\usepackage[english]{babel}
\usepackage[
paperheight=8.5in,
paperwidth=5.5in,
outer=0.5in,
inner=0.6in,
bottom=1in,
top=0.7in
]{geometry}

\setlength{\parindent}{0em}

\usepackage{xcolor}
\newcommand{\todo}[1]{{\color{red}{TODO: #1}}}
\newcommand{\proofincomplete}{\todo{Finish proof}}

\usepackage[backend=biber]{biblatex}
\addbibresource{bibliography.bib}

\usepackage{imakeidx}
\makeindex[intoc=true]

%\usepackage{glossaries-extra}
%\makeglossaries
%\input{chapters/chapter-2/chapter-2-glossary.tex}

\usepackage{amsmath}
\usepackage{amsfonts}
\usepackage{amssymb}
\usepackage{amsthm}
\usepackage{mathrsfs}

\theoremstyle{definition}
\newtheorem{theorem}{Theorem}[chapter]
\newtheorem*{corollary}{Corollary}
\newtheorem{definition}[theorem]{Definition}
\newtheorem{pproposition}[theorem]{Proposition}
\newtheorem{exercise}{Exercise}[chapter]
\newtheorem*{proposition}{Proposition}
\newtheorem*{solution}{Solution}
\newtheorem*{remark}{Remark}

\usepackage{hyperref}

\title{Solutions Notebook for\\Principles of Mathematical Analysis\\by Walter Rudin}
\author{Jose Fernando Lopez Fernandez}
\date{17 March, 2020 -- \today}

\begin{document}
	\frontmatter
	\maketitle
	\tableofcontents
	\mainmatter
%	\chapter{Construction of the Real Numbers}

%	\chapter{Properties of the Real Numbers}

%	\chapter{Limits and Continuity}

%	\chapter{Differentiation}

%	\chapter{Integration}

%	\chapter{Limits to Infinity}

%	\chapter{Transcendental Functions}

%	\chapter{Sequences}

%	\chapter{Series}

%	\chapter{Sequences and Series of Functions}

%	\input{chapters/chapter-11.tex}
%	\appendix
%	\input{chapters/appendix-2.tex}
%	\input{chapters/appendix-3.tex}
%	\backmatter
%	\printbibliography[heading=bibintoc]
%	\printglossaries
%	\printindex
\end{document}
